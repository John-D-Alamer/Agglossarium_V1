\documentclass[1pt, onecolumn, oneside, a4paper] {book}
\usepackage[utf8] {inputenc} % Encodage des caractères
\usepackage[french] {babel} % typographie française
\usepackage[T1] {fontenc} % encodage des polices
\usepackage{xspace} % pour gérer les espaces 
\usepackage{txfonts} % polices pdf de qualité
\usepackage{verse}% pour écrire des verses
\usepackage{listings}
\usepackage{array,multirow,makecell}
\usepackage{CJKutf8}
\setcellgapes{1pt}
\makegapedcells
\usepackage{supertabular} % tableaux qui tiennent sur plusieurs pages

\newenvironment{verse01}{
\fontfamily{jkplos}\selectfont

}

\newcommand\x{\\[\stanzaskip]} 



\addtolength{\stanzaskip}{5pt} % rajoute 10 points



%index
\usepackage{makeidx}
\makeindex

%césure à la française
\newlength{\phila}
\newcommand\phicesure[2]{%
\settowidth\phila{#1}\addtolength{\phila}{-\vindent}
{#1}\verselinebreak\makebox[\phila][r]{[#2}}

% mise en page
\pagestyle{plain} %numérotation des pages en bas 
\usepackage[inner=2cm, top =4cm, outer=2cm, bottom=4cm] {geometry} %regler les marges





%en tête 
\usepackage{fancyhdr}
\pagestyle{fancy}
\usepackage{lastpage}
\renewcommand\headrulewidth{0,2pt}
\fancyhead[L]{{ANNEXE $\alpha$ : bibliographie relative à l'Agglossarium}}
\fancyhead[R]{{ page \thepage/\pageref{LastPage}}}

%pied de page  
\renewcommand\footrulewidth{0,2pt}
\fancyfoot[C]{ \textbf{ Version 1.0}\\ \textit{ (dernière compilation avec Texmaker : \today)}
}
\fancyfoot[R]{}

\def\cache{\def\?##1{...}}
\def\visible{\def\?##1{##1}}



\begin{document}
\begin{center}


\thispagestyle{empty}
{\fontsize{30}{48}\selectfont ANNEXE $\alpha$ \\ 
\vspace{8mm}
\Large (V.1.0)\\ 
\vspace{5mm}
\Huge\fontfamily{jkplos}\selectfont
{\textit{Bibliographie relative\\ à l'Agglossarium}}}\\
\vspace{15cm}
\vspace{\fill}
\end{center}
%************************************************************

\newpage



	




 \tablefirsthead{\hline
Textes & Autairz & Publication originale (Titre, Éditeur, Date) & ITraductairz   & Traduction (Titre, Éditeur, Date)\\}
\tablehead{\hline
Textes & Autairz & Publication originale (Titre, Éditeur, Date) & Traductairz   & Traduction (Titre, Éditeur, Date) \\}
\tabletail{\hline}
% \multicolumn{5}{|r|}{\sl Continue à la page suivante}\\ \hline
\fontfamily{jkplos}\selectfont
\begin{verse01}
\begin{supertabular}{|>{\centering\arraybackslash} p{0,8cm} |>{\centering\arraybackslash} p{3cm} |>{\centering\arraybackslash} p{4.5cm} |>{\centering\arraybackslash} p{2cm} |>{\centering\arraybackslash} p{4.5cm}|} \hline



A01 &
	Mathon, Bernard &
	«\,Un vamasur nommé palisir\,», \textit{in L’Hexagone halluciné}, 
	Le livre de poche, 1988 && \\
	
\hline
A02 &
	Jemesin, Nora K &
	\textit{The Stone Sky}, Orbit, 2017 &
	Charrier, Michelle &
	\textit{Les livres de la terre fracturée}, J’ai Lu, 2018  \\
 
 \hline
A03 &
	Le Guin, Ursula K &
	\textit{Always Coming Home} Harper \& Row, 1985 &
	Reinharez, Isabelle &
	\textit{La vallée de l’éternel retour,}, Mnémos, 2019  \\
	
\hline
A04 &
	Butler, Octavia &
	\textit{Parable of the talents}, Seven stories press, 1998 &
	Tate, Iawa &
	\textit{La parabole des talents},Au diable vauvert, 2001 \\
	
\hline
A05 &
	Anderson, Poul & 
	«\,Sam Hall\,», in \textit{Astounding Science Fiction n° 273} , John W. Campbell 	, 1953 &
	Brèque, Jean-Daniel &
	«\,Sam Hall\,», in \textit{ Le chant du barde, les meilleurs récits de Poul Anderson, Le 	Bélial, 2001} \\

\hline
A06 &
	Anderson, Poul &
	\textit{Goat Song,} Mercury press, 1972 &
	Brèque, Jean-Daniel &
	«\,Le chant du barde\,», in \textit{Le chant du barde}, les meilleurs récits de Poul Anderson, Le Bélial, 2001 \\

\hline 
A07 &
	Ayerdhal &
	\textit{Parleur ou les Chroniques d’un rêve enclavé}, J’ai Lu, 1997 &&\\
	
\hline
A08 &
	Boulle, Pierre &
	\textit{Les jeux de l’esprit}, J’ai Lu, 1971 &&\\
	
\hline
A09 &
	Le Guin, Ursula K &
	\textit{Always Coming Home}, Harper \& Row, 1985 &
	Reinharez, Isabelle &
	\textit{La vallée de l’éternel retour}, Mnémos, 2019 \\
	
\hline
A10 &
	Delany, Samuel R &
	\textit{Babel 17}, Ace Books, 1966 &
	Perrin, Mimi &
	\textit{Babel 17}, Calmann-Lévy, 1973 \\
	
\hline
A11 &
	Klein, Gérard &
	«\,Mémoire vive, mémoire morte\,», in \textit{Mémoire vive, mémoire morte}, Robert Laffont, 2011 & &\\
	
\hline
A12 &
	Damasio, Alain &
	\textit{Les furtifs}, La Volte, 2019 &&\\
	
\hline
A13 &
	Le Guin, Ursula K &
	\textit{Always Coming Home}, Harper \& Row, 1985 & 
	Reinharez, Isabelle &
	\textit{La vallée de l’éternel retour}, Mnémos, 2019 \\
	
\hline
A14 &
	Demuth, Michel &
	«\,Nocturnes pour démon\,», in \textit{À l’est du cygne}, La Bélial, 2010 && \\
	
\hline 
A15 &
	Anderson, Poul &
	\textit{The graveyard earth}, Ziff-Davis Publishing Company, 1964 &
	Wiznitze, Martine &
	«\,Le coeur funéraire\,», in \textit{Une rose pour l’Ecclésiaste}, J’ai lu, 1980 \\
	
\hline
A16 &
	Anderson, Poul &
	\textit{The graveyard earth}, Ziff-Davis Publishing Company, 1964 &
	Wiznitze, Martine &
	«\,Le coeur funéraire\,», in \textit{Une rose pour l’Ecclésiaste}, J’ai lu, 1980 \\
	
\hline
A17 &
	Minard, Céline &
	\textit{Le dernier monde}, Denoël, & & \\ 

\hline
B01 &
	Wyndham, John &
	«\,Survival\,», in 	\textit{Thrilling Wonder Stories}, Experimenter Publishing, 1964 &&\\


\hline
B02 &
	Jemesin, Nora K &
	\textit {The Stone Sky}, Orbit, 2017 &
	Charrier, Michelle &
	\textit{Les livres de la terre fracturée}, J’ai Lu, 2018 \\
	
\hline
B03 &
	Banks, Iain &
	\textit{The state of the art}, Mark V. Ziesing, 1989 &
	Quemener, Sonia &
	\textit{Une rose pour l’Ecclésiaste}, J’ai lu, 1980\\
	
\hline
C01 &
	Banks, Iain &
	\textit{The state of the art}, Mark V. Ziesing, 1989 &
	Quemener, Sonia &
	\textit{Une rose pour l’Ecclésiaste}, J’ai lu, 1980\\
	
\hline
C02 &
	Anderson, Poul &
	\textit{Goat Song}, Mercury press, 1972 &
	Brèque, Jean-Daniel &
	«\,Le chant du barde\,», in \textit{Le chant du barde}, les meilleurs récits de Poul Anderson, Le Bélial, 2001 \\
	
\hline
C03 &
	Le Guin, Ursula K &
	\textit{Always Coming Home}, Harper \& Row, 1985 &
	Reinharez, Isabelle &
	\textit{La vallée de l’éternel retour}, Mnémos, 2019 \\
	
\hline
C04 &
	Orwell, Georges &
	\textit{1984}, Secker and Warburg, 1949 &
	Audiberti, Amélie &
	1984, Gallimard, 1950 \\
	
\hline 
C06 & 
	Mayer, Bernadette &
	\textit{Utopia}, United artists books, 1984 &
	Caro, Jean-François &
	\textit{Utopia}, Future, 2016 \\
	

\hline 
C07 &
	Damasio, Alain &
	\textit{La horde du contrevent}, La volte, 2004 &&\\
	
\hline 
C08 & 
	Acker, Kathy &
	\textit{Empire of the senseless}, Grove press, 1988 &&\\
	
\hline
C09 &
	Panchard, Georges &
	«\,Stellarum nox\,», in \textit{Les horizons divergents}, Le livre de poche, 1999 &&\\
	
\hline 
C10 &
	Minard, Céline &
	\textit{Le dernier monde}, Denoël, 2007 &&\\
	
\hline 
C11 &
	Lafferty, Raphaël A &
	\textit{Space chantey}, Ace books, 1968 &
	Perrin, Mimi &
	\textit{Les chants de l’espace}, Opta, 1974 \\
	
\hline
D01 &
	Zelazny, Roger &
	\textit{The graveyard earth}, Ace Books, 1964 & 
	Deutsch, Michel &
	«\,Le coeur funéraire\,», in Une rose pour l’Ecclésiaste, J’ai lu, 1980\\
	
\hline 
D03 &
	Cixin, Liu &
	 \begin{CJK}{UTF8}{min}
	死神永生
    \end{CJK}, Chongqing Press, 2010 &
	Gaffric, Gwennaël &
	\textit{La mort immortelle}, Actes Sud, 2018 \\
	
\hline
D04 &
	Anderson, Poul &
	\textit{Goat Song}, Mercury press, 1972 &
	Brèque, Jean-Daniel &
	«\,Le chant du barde\,», in \textit{Le chant du barde}, les meilleurs récits de Poul Anderson, Le Bélial, 2001 \\
		
\hline
D05
	Mayer, Bernadette &
	\textit{Utopia}, United artists books, 1984 & 
	Caro, Jean-François &
	\textit{Utopia}, Future, 2016 \\
	
\hline
D06 &
	Fontenay, Charles &
	\textit{The silk and the song}, Mercury Press, 1956 &
	Durand, Roger &
	\textit{La soie et la chanson}, Opta, 1957 \\
	
	
\hline
E01 &
	Jemesin, Nora K &
	\textit{The fifth season}, Orbit, 2015 &
	Charrier, Michelle &
	\textit{La cinquième saison}, J’ai Lu, 2017 \\
	
\hline 
E02 &
	Butler, Octavia E &
	\textit{Parable of the sower}, Four walls eight windows, 1993 &
	Rouard, Philippe &
	\textit{La parabole du semeur}, J’ai lu, 1995 \\
	
\hline 
E03 &
	Nearing, Homer &
	\textit{The Poetry Machine}, Mercury Press, 1950 & &
	«\,La machine à poésie\,», in \textit{La Grande anthologie de la science fiction}, Le livre de poche, 1974 \\

\hline
E04 &
	Herbert, Frank &
	\textit{Dune}, Chilton books, 1965 &
	Demuth, Michel &
	\textit{Dune}, Robert Laffont, 1970 \\
	
\hline 
E05 &
	Cixin, Liu &
	\begin{CJK}{UTF8}{min}
	黑暗森林, Chongqing Press, 2008
	\end{CJK} &
	Gaffric, Gwennaël &
	\textit{La forêt sombre}, Actes Sud, 2017 \\
	
\hline
E06 &
	Sturgeon, Theodore &
	«\,A Saucer of Loneliness\,» dans \textit{Galaxy Science Fiction} n. 27, 1953 &
	Straschitz, Frank &
	\textit{Le disque de solitude}, Nuit et jour, 1956\\
	

\hline 
E07 &
	Wittig, Monique &
	\textit{Les guérillères}, Éditions de Minuit, 1969 &&\\
	
\hline 
E08 &
	Heinlein, Robert &
	\textit{The Green Hills of Earth}, 1947 &
	Billon, Pierre &
	«\,Les Vertes Collines De La Terre\,», dans Histoire du Futur, Opta, 1967 \\
	
\hline 
E09 &
	Wolfe, Gene &
	\textit{The Death of Doctor Island}, Random House, 1973 &
	Planchat, Henry-Luc &
	«\,La Mort du Docteur Ile\,», dans \textit{L’île du Docteur Mort et autres histoires}, Robert Laffont, 1980\\
	
\hline
E10 &
	Silverberg, Robert &
	\textit{Road to Nightfall}, Fantastic Universe 1953 &
	Hupp, Philippe &
	«\,Le Chemin de la Nuit\,», \textit{Histoires de Survivants}, Le Livre de Poche, 1983\\
	
\hline
E11 &
	Fontenay, Charles &
	\textit{The Silk and the Song}, Mercury press, 1956 &
	Durand, Roger &
	\textit{La soie et la chanson}, Opta, 1957 \\
	
\hline
E12 &
	Walther, Daniel &
	\textit{Est-ce moi qui blasphème ton nom, seigneur ?}, Opta, 1973 &\\
	
\hline 
F01 &
	Wittig, Monique &
	\textit{Les guérillères}, Éditions de Minuit, 1969 & \\
	

\hline
F02 &
	Matheson, Richard &
	\textit{Dance of the Dead}, Ballantine Books, 1961 &
	Martin, Bruno &
	\textit{Danse Macabre}, Opta, 1983 \\
	
\hline
F03 &
	Orwell, Georges &
	\textit{1984}, Secker and Warburg, 1949 &
	Audiberti, Amélie &
	\textit{1984}, Gallimard, 1950 \\
	
\hline 
H01 &
	Le Guin, Ursula K &
	\textit{Always Coming Home}, Harper & Row, 1985
	Reinharez, Isabelle &
	\textit{La vallée de l’éternel} retour, Mnémos, 2019 \\
	
\hline 
H02 &
	Fontenay, Charles &
	\textit{The Silk and the Song}, Mercury press, 1956 &
	Durand, Roger &
	\textit{La soie et la chanson}, Opta, 1957 \\
	
\hline
I01 &
	Nasir, Jamil &
	\textit{Tower of dreams}, Spectra, 1999 &
	Haas, Dominique &
	\textit{La tour des rèves}, Pocket, 2001 \\
	
\hline
I02 &
	Ayerdhal &
	\textit{Parleur ou les Chroniques d’un rêve enclavé}, J’ai Lu, 1997 &\\
	
\hline
I03 &
	Delany, Samuel R &
	\textit{Babel 17}, Ace Books, 1966 &
	Perrin, Mimi &
	\textit{Babel 17}, Calmann-Lévy, 1973 \\
	
\hline
I04 &
	Adaf, Shimon &
	\textit{Like a coin entrusted in faith}, Apex book company, 2015 &&\\
	
\hline
I05 &
	Herbert, Frank &
	«\,The heaven makers\,», dans \textit{Amazing Stories}, Experimenter Publishing, 1967 &\\
	
\hline
I06 &
	Le Guin, Ursula K &
	\textit{Always Coming Home}, Harper \& Row, 1985 &
	Reinharez, Isabelle &
	\textit{La vallée de l’éternel retour}, Mnémos, 2019 \\
	
\hline
I07 &
	Butler, Octavia &
	\textit{Parable of the talents}, Seven stories press, 1998 &
	Tate, Iawa &
	\textit{La parabole des talents}, Au diable vauvert, 2001 \\
	
\hline
J01 &
	Sturgeon, Theodore &
	\textit{The World Well Lost}, Abelard Press, 1953 &
	Martin, Bruno &
	«\,Monde interdit\,», dans \textit{Les songes superbes }de Theodore Sturgeon, Casterman, 1978\\
	
\hline 
J02 &
	Cixin, Liu &
	\begin{CJK}{UTF8}{min}
	死神永生
	\end{CJK} , Chongqing Press, 2010 &
	Gaffric, Gwennaël &
	\textit{La mort immortelle}, Actes Sud, 2018 \\
	
\hline 
J03 &
	Delany, Samuel R &
	\textit{The Einstein Intersection}, Ace Books, 1967 &&
	\textit{L’intersection Einstein}, Opta, 1977\\
	
\hline 
J04 &
	Dunyach, Jean-Claude &
	\textit{Étoiles mortes}, J’ai lu, 2000 &\\
	
\hline 
J05 &
	Banks, Iain &
	\textit{Descendant, in The state of the art}, Mark V. Ziesing, 1989 &
	Quemener, Sonia &
	«\,Descente\,», dans \textit{L’essence de l’art}, Le bélial, 2010\\
	
\hline 
J06 &
	Herbert, Frank &
	\textit{Dune}, Chilton books, 1965 &
	Demuth, Michel &
	\textit{Dune}, Robert Laffont, 1970 \\
	
\hline 
J07 & 
	Vonnegut, Kurt &
	\textit{Welcome to the Monkey House}, Playboy enterprises, 1968 &
	Polanis, Jacques &
	\textit{Bienvenue au pavillon des singes}, Le Livre de Poche, 1978 \\
	
\hline 
J08 &
	Lafferty, Raphaël A &
	\textit{Nor limestones islands}, Ace Books, 1971 &
	Canet, Charles &
	\textit{Ni les îles de calcaire qui volent dans le ciel}, Le Livre de Poche, 1984 \\
	
\hline 
J09 & 
	Minard, Céline &
	\textit{Le dernier monde}, Denoël, 2007 & &\\

\hline
J10 & 
	Heinlein, Robert &
	\textit{The Green Hills of Earth}, The Saturday Evening Post, 1947 &
	Billon, Pierre &
	«\,Les Vertes Collines De La Terre\,», dans \textit{Histoire du Futur}, Opta, 1967\\
	

\hline 
J11 &
	Russ, Joanna &
	The Female Man, Batam Books, 1975 &
	Planchat, Henry-Luc &
	\textit{L’autre moitié de l’homme}, Robert Laffont, 1977 \\
	
\end{supertabular}
\end{verse01}

%************************************************************


\newpage
\center
\thispagestyle{empty}
\textit{L'Agglossarium} a été mis en page avec \LaTeX\\ via l'éditeur de document Texmaker.\\ 

\medbreak
\medbreak
Ont été utilisés comme \textit{package} :

\{inputenc\} pour l'encodage des caractères.\\
 \{babel\} pour la  typographie française.\\
\{fontenc\} pour l'encodage des polices.\\
\{xspace\} pour pour gérer les espaces.\\
\{txfonts\} pour générer une polices pdf de qualité.\\
\{verse\} pour pour écrire des verses.\\

\medbreak\medbreak
La commandes : 

\{phicesure\} a été créée pour franciser\\  le système de césures du package \{verse\}.

\medbreak\medbreak
L'environnement  : 

\{verse01\} a été créé pour plus de souplesse\\ sur la gestion des vers.

\medbreak\medbreak
PDF et code source disponible sur \\ 
http://editionsburnaout.fr/

\medbreak\medbreak
ISBN : 9782493534026





%************************************************************
\end{document}
